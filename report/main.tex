\documentclass[a4paper]{jlreq}

\usepackage{graphicx}
\usepackage{listings}
\lstset{
  basicstyle=\small\ttfamily,
  identifierstyle=\small,
  ndkeywordstyle=\small,
  tabsize=4,
  frame={shadowbox},
  frameround={ffff},
  breaklines=true,
  columns=[l]{fullflexible},
  numbers=left,
  numbersep=5pt,
  numberstyle=\scriptsize,
  stepnumber=1,
  lineskip=-0.5ex
}
\renewcommand{\lstlistingname}{ソースコード} % キャプション名の変更
\usepackage{hyperref}
\usepackage{here}

\title{主専攻実験 分散プログラミング \\課題1:リングアルゴリズム}
\author{201811377 広瀬智之}
\date{\today}

\begin{document}

\maketitle

\section{概要}

別々のノードにあるプロセスをリング状に繋ぎ、途中からプロセスを追加・脱退させることを可能にする。

またコーディネータと呼ぶノードを選出しハートビートを行うことで、繋がれたプロセスの生存を確認する。
さらにコーディネータからのハートビートが途絶えたら他のプロセスから新たなコーディネータを選出するように実装する。

\section{アルゴリズム}

この章では実装に用いるアルゴリズムを解説する。

\subsection{プロセスの繋ぎ方}

プロセスを繋ぐアルゴリズムには双方向リンクリストを用いる。


\section{実装}

プログラムは\ref{sec:proglist}章に掲載する。



\section{プログラムリスト}
\label{sec:proglist}

\lstinputlisting[label=p1,language=c,caption=main.c]{../src/main.c}
\lstinputlisting[label=p2,language=c,caption=rpc.c]{../src/rpc.c}
\lstinputlisting[label=p3,language=c,caption=ring.c]{../src/ring.c}

以下はヘッダファイルである。

\lstinputlisting[label=p4,language=c,caption=main.h]{../src/main.h}
\lstinputlisting[label=p5,language=c,caption=rpc.h]{../src/rpc.h}
\lstinputlisting[label=p6,language=c,caption=ring.h]{../src/ring.h}

\end{document}
